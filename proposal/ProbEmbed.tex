\documentclass[12pt]{article}
\usepackage{amsmath}
\usepackage{amssymb}
\usepackage{amsthm}
\usepackage{amscd}
\usepackage{amsfonts}
\usepackage{graphicx}%
\usepackage{fancyhdr}
\usepackage{hyperref}
\usepackage{titlesec}
\usepackage{titling}
\usepackage{stmaryrd}
\usepackage[margin=2cm]{geometry}

\theoremstyle{plain} \numberwithin{equation}{section}
\newtheorem{theorem}{Theorem}[section]
\newtheorem{corollary}[theorem]{Corollary}
\newtheorem{conjecture}{Conjecture}
\newtheorem{lemma}[theorem]{Lemma}
\newtheorem{proposition}[theorem]{Proposition}
\theoremstyle{definition}
\newtheorem{definition}[theorem]{Definition}
\newtheorem{finalremark}[theorem]{Final Remark}
\newtheorem{remark}[theorem]{Remark}
\newtheorem{example}[theorem]{Example}
\newtheorem{question}{Question} 

\setlength{\parskip}{0.5 \baselineskip}%
\setlength{\headheight}{15pt}
\setlength{\droptitle}{-100pt}

\titlespacing{\paragraph}{%
  0pt}{%              left margin
  0pt}{% space before (vertical)
  1em}%               space after (horizontal)

\pagestyle{fancy}\lhead{September 2014} \rhead{ProbEmbed}
\chead{{\large{Proposal}}} \lfoot{} \rfoot{\bf \thepage} \cfoot{}

\newcounter{list}

\title{Learning embeddings with Dirichlet processes and heavy tails}
\author{Adith Swaminathan, \href{fa234@cornell.edu}{fa234@cornell.edu} \\
Chenhao Tan, \href{chenhao@cs.cornell.edu}{chenhao@cs.cornell.edu} \\
Moontae Lee, \href{ml2255@cornell.edu}{ml2255@cornell.edu}}

\begin{document}
\maketitle

\begin{question}{What are you trying to do?}
\end{question}
We are developing a generative model of word embeddings that
\begin{itemize}
\item Takes as input $D$: all observed text
\item Outputs an embedding $X$: words situated in a vector space
\item such that the embeddings \emph{cluster around concepts}
\item distances/directions in the embedding space directly model data distribution
\item which is scalable to learn
\item and also recover good performance of existing embeddings in downstream text analysis tasks.
\end{itemize}

\begin{question}{Why is it hard?}
\end{question}


\begin{question}{How is it being done today?}
\end{question}

\begin{question}{What has changed that makes it possible to do something else?}
\end{question}

\begin{question}{How will you know whether you are doing a good job?}
\end{question}

\begin{question}{Who will want to know what your results are?}
\end{question}

\bibliographystyle{alpha}
\bibliography{ProbEmbed}

\end{document}
