\documentclass[letterpaper]{article}
\usepackage{amsmath}
\usepackage{amssymb}
\usepackage{amsthm}
\usepackage{amscd}
\usepackage{amsfonts}
\usepackage{graphicx}
\usepackage{fancyhdr}
\usepackage{hyperref}
\usepackage{titlesec}
\usepackage{titling}
\usepackage{stmaryrd}
\usepackage{enumitem}
\usepackage[margin=2.5cm]{geometry}
\usepackage{tikz}
\usetikzlibrary{bayesnet}


\theoremstyle{plain} \numberwithin{equation}{section}
\newtheorem{theorem}{Theorem}[section]
\newtheorem{corollary}[theorem]{Corollary}
\newtheorem{conjecture}{Conjecture}
\newtheorem{lemma}[theorem]{Lemma}
\newtheorem{proposition}[theorem]{Proposition}
\theoremstyle{definition}
\newtheorem{definition}[theorem]{Definition}
\newtheorem{finalremark}[theorem]{Final Remark}
\newtheorem{remark}[theorem]{Remark}
\newtheorem{example}[theorem]{Example}
\newtheorem{question}{Question}

\setlength{\parskip}{0.5 \baselineskip}%
\setlength{\headheight}{15pt}
\setlength{\droptitle}{-100pt}

\titlespacing{\paragraph}{%
  0pt}{%              left margin
  0pt}{% space before (vertical)
  1em}%               space after (horizontal)

% math notations
\newcommand{\hyper}{H^*}
\newcommand{\nd}{n_d}
\newcommand{\htd}[2]{\frac{1}{1+||#1-#2||^2}}
\newcommand{\htdnorm}[1]{Z(#1)}
\newcommand{\Xwi}{X_{w_i}}
\newcommand{\muzi}{\mu_{z_i}}
\newcommand{\norm}[1]{||#1||^2}
\newcommand{\tsup}[2]{#1^{#2}}
\DeclareMathOperator*{\argmin}{arg\,min}
\DeclareMathOperator*{\argmax}{arg\,max}


\title{When word embedding encounters topic modeling}
\author{%
  Moontae Lee \\ \href{ml2255@cornell.edu}{ml2255@cornell.edu}
  \and Adith Swaminathan\\ \href{fa234@cornell.edu}{fa234@cornell.edu}
  \and Chenhao Tan\\ \href{chenhao@cs.cornell.edu}{ct425@cornell.edu} 
}


\begin{document}

\maketitle

Figure \ref{fig:tw_model} shows a model that combines word embeddings with topic modeling. The right hand side of this model is exactly the same as normal LDA. What Adith and I figured out this afternoon is that using Chinese restaurant processes to control the number of clusters will have to involve hierarchical Dirichlet processes. Thus we think it is a good idea to start with simpler setting. Thus we have a fixed number of topics/concept clusters ($K$).
The left side of the model is a mixture of Gaussian. Therefore the full model is

\begin{align}
Pr(m, X, \pi, \mu, z, \theta, D|\hyper) & = Pr(D|m,X,\mu,z,\theta,\hyper)Pr(z|\theta, \hyper) Pr(\theta|\hyper) \nonumber\\
& Pr(X|m,\mu,\hyper) Pr(m|\pi), Pr(\mu|\hyper) \nonumber\\
& \propto \prod_{d\in D}\prod_{i=1}^{\nd}\frac{\htd{\Xwi}{\muzi}}{\htdnorm{\muzi}} \prod_{d\in D}\prod_{i=1}^{\nd} \theta_{dz_i} \prod_{d} Dir(\alpha) \nonumber \\
& \prod_{x \in V} \pi_{m_x} \frac{e^{-||x-\mu_{m_x}||^2/2\sigma^2}}{\sqrt{2\pi}\sigma} \prod_{k=1}^K \frac{1}{\sqrt{2\pi} \rho} e^{||\mu_k||^2/2\rho^2}
\end{align}


It is $\propto$ because we assumed uniform norm on $\pi$. We tried to write out conditional distributions for Gibbs sampler to make ourselves familiar with the model. But this model clearly cannot be solved efficiently by using Gibbs sampling. 

\begin{figure}[ht]
\begin{center}
% Modified from model_lda.tex
%
% Copyright (C) 2010,2011 Laura Dietz
% Copyright (C) 2012 Jaakko Luttinen
%
% This file may be distributed and/or modified
%
% 1. under the LaTeX Project Public License and/or
% 2. under the GNU General Public License.
%
% See the files LICENSE_LPPL and LICENSE_GPL for more details.

% Topic model with word embeddings

\begin{tikzpicture}[x=1.7cm,y=1.8cm]

  % Nodes

  % topic part
  \node[obs]                   (w)      {$w$} ; %
  \node[latent, above=of w]    (z)      {$z$} ; %
  \node[latent, above=of z]    (theta)  {$\theta$}; %
  \node[latent, above=of theta] (alpha) {$\alpha$};

  % mixture gaussian for word embeddings

  % More nodes
  \node[latent, left=of w] (X)  {$X$}; %
  \node[latent, above=of X]  (mu) {$\mu$}; %
  \node[latent, above=of mu]  (rho) {$\rho$}; %
  \node[latent, left=of X]  (m) {$m$}; %
  \node[latent, left=of m]  (pi) {$\pi$}; %
  \node[latent, above=of m]  (sigma) {$\sigma$}; %

  \edge {alpha} {theta} ; %
  \edge {theta} {z} ; %
  \edge {z} {w} ; %
  \edge {mu} {w} ; %
  \edge {X} {w} ; %
  \edge {rho} {mu} ; %
  \edge {mu} {X} ; %
  \edge {m} {X} ; %
  \edge {sigma} {X} ; %
  \edge {pi} {m} ; %

  \plate [inner sep=5pt] {words} { %
    (z)
    (w)
  } {$n_d$}; %
  \plate {documents} { %
    (words) %
    (theta) %
  } {$D$} ; %
  \plate {vocab} { %
    (m)
    (X) %
  } {$|V|$} ; %
  \plate {concepts} { %
    (mu)
  } {$K$} ; %


\end{tikzpicture}


\end{center}
\caption{Topic modeling with word embedding.\label{fig:tw_model}}
\end{figure}


So we started to derive variational methods for this. We start with EM algorithm of the mixture of Gaussian part with prior.
We take the left part related to mixture of Gaussian first.
The posterior distribution that we want to maximize is

\begin{align}
Pr(\pi, \mu|X, \hyper) & = \sum_m Pr(m, \mu, \pi|X, \hyper) \nonumber\\
Pr(m, \mu, \pi|X, \hyper) &= \prod_{i=1}^V \pi_{m_i} e^{-\norm{\mu_{m_i}-X_i}/2\sigma^2}\prod_{k=1}^K e^{-\norm{\mu_k}/2\rho^2} \nonumber\\
\log{Pr(m, \mu, \pi|X, \hyper)} &= \sum_{i=1}^{V}(log\pi_{m_i} - \norm{\mu_{m_i}-X_i}/2\sigma^2)-\sum_{k=1}^K \norm{\mu_k}/2\sigma^2 + C \nonumber
\end{align}

E-step: we compute $E_{m|\tsup{\pi}{t}, \tsup{\mu}{t}}\log{Pr(m, \mu, \pi|X, \hyper)}$.

What we need is 

$$\gamma_{ik}=Pr(m_i=k|\tsup{\pi}{t}, \tsup{\mu}{t})=\frac{\tsup{\pi_k}{t}e^{-\norm{\tsup{\mu_k}{t}-X_i}/2\sigma^2}}{\sum_{k'}\tsup{\pi_{k'}}{t}
e^{-\norm{\tsup{\mu_{k'}}{t}-X_i}/2\sigma^2}}$$

M-step: we update parameters so that $\tsup{\pi}{t+1}, \tsup{\mu}{t+1} = \argmax_{\tsup{\pi}{t+1}, \tsup{\mu}{t+1}}E_{m|\tsup{\pi}{t}, \tsup{\mu}{t}}\log{Pr(m, \mu, \pi|X, \hyper)}$

\begin{align}
E_m \log{Pr(m, \mu, \pi|X, \hyper)} &= \sum_{i=1}^V\sum_{k=1}^K \gamma_{ik}\log{\pi_k} - \sum_{i=1}^V\sum_{k=1}^K \gamma_{ik} \norm{\mu_k - X_i}/2\sigma^2 - \sum_{k=1}^K \norm{\mu_k}/2\rho^2 \nonumber \\
&= \sum_{k=1}^K \log{\pi_k} \sum_{i=1}^V \gamma_{ik} + \sum_{k=1}^K (\norm{\mu_k}/2\rho^2 + \sum_{i=1}^V \gamma_{ik}\norm{\mu_k-X_i}/2\sigma^2) \nonumber
\end{align}

The first part is minimize cross entropy, using Lagrange multiplier, we can get that 
$$\pi_k = \sum_{i=1}^V \gamma_{ik}/V$$

The second part related to $\mu_k$ can be solved in closed form as well

$$\mu_k = \frac{\sum_{i=1}^V\gamma_{ik}X_i/\sigma^2}{V/\sigma^2+K/\rho^2}$$

\end{document}

%%% Local Variables: 
%%% mode: tex-pdf
%%% TeX-master: t
%%% End: 
